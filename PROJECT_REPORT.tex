\documentclass[12pt,a4paper]{article}


\usepackage[utf8]{inputenc}
\usepackage[T1]{fontenc}
\usepackage[margin=1in]{geometry}
\usepackage{setspace}
\usepackage{graphicx}
\usepackage{hyperref}
\usepackage{listings}
\usepackage{xcolor}
\usepackage{titlesec}
\usepackage{fancyhdr}
\usepackage{enumitem}
\usepackage{booktabs}
\usepackage{caption}
\usepackage{tcolorbox}
\usepackage{microtype}

\onehalfspacing

\definecolor{primary}{RGB}{0,70,140}
\definecolor{secondary}{RGB}{0,120,200}
\definecolor{lightgray}{RGB}{245,245,245}
\definecolor{codegreen}{rgb}{0,0.6,0}
\definecolor{codegray}{rgb}{0.5,0.5,0.5}
\definecolor{codepurple}{rgb}{0.58,0,0.82}
\definecolor{backcolour}{rgb}{0.97,0.97,0.97}

\lstdefinestyle{mystyle}{
    backgroundcolor=\color{backcolour},
    commentstyle=\color{codegreen},
    keywordstyle=\color{primary},
    numberstyle=\tiny\color{codegray},
    stringstyle=\color{codepurple},
    basicstyle=\ttfamily\footnotesize,
    breaklines=true,
    captionpos=b,
    keepspaces=true,
    numbers=left,
    numbersep=6pt,
    showspaces=false,
    showstringspaces=false,
    showtabs=false,
    tabsize=2,
    frame=single,
    rulecolor=\color{lightgray}
}

\lstset{style=mystyle}


\pagestyle{fancy}
\fancyhf{}
\lhead{\footnotesize CSE731: Software Testing}
\rhead{\footnotesize Client-Side Bypass Testing}
\rfoot{\footnotesize Page \thepage}


\titleformat{\section}
  {\normalfont\Large\bfseries\color{primary}}
  {\thesection}{1em}{}

\titleformat{\subsection}
  {\normalfont\large\bfseries\color{secondary}}
  {\thesubsection}{1em}{}

\titleformat{\subsubsection}
  {\normalfont\normalsize\bfseries}
  {\thesubsubsection}{1em}{}

\setlist[itemize]{left=1.5em}
\setlist[enumerate]{left=1.5em}



\hypersetup{
    colorlinks=true,
    linkcolor=primary,
    filecolor=magenta,
    urlcolor=secondary,
    citecolor=primary
}

% ----------------------------------------------------------------------
% Tcolorbox styles
% ----------------------------------------------------------------------
\tcbset{
    colback=white,
    colframe=primary,
    coltitle=primary,
    fonttitle=\bfseries,
    arc=1.5mm
}

\newtcolorbox{abstractbox}{
    colback=lightgray,
    colframe=primary,
    title=Abstract,
    boxrule=0.5pt,
    left=6pt,right=6pt,top=6pt,bottom=6pt
}

\newtcolorbox{keybox}{
    colback=lightgray,
    colframe=secondary,
    boxrule=0.4pt,
    left=6pt,right=6pt,top=6pt,bottom=6pt
}

% ----------------------------------------------------------------------
% Document
% ----------------------------------------------------------------------
\begin{document}

% ----------------------------------------------------------------------
% Title page
% ----------------------------------------------------------------------
\begin{titlepage}
    \centering
    \vspace*{1.5cm}

    {\Large International Institute of Information Technology Bangalore\par}
    \vspace{0.3cm}
    {\large Department of Computer Science and Engineering\par}

    \vspace{2.2cm}

    {\Huge\bfseries Client-Side Bypass Testing of a Taxi Booking Web Application\par}

    \vspace{1cm}
    {\Large\textbf{CSE731: Software Testing}\par}
    \vspace{0.1cm}
    {\large Term I, 2025--26\par}

    \vspace{2cm}

    \begin{tcolorbox}[colback=white,colframe=primary,boxrule=0.8pt,
        width=0.7\textwidth,center title,title=Project Category]
        \centering
        Client-Side Web Application Bypass Testing
    \end{tcolorbox}

    \vspace{2cm}

    {\large\textbf{Submitted by}\par}
    \vspace{0.6cm}
    \begin{tabular}{rl}
        \textbf{MT2024106} & Omkar Dhananjay Satav \\
        \textbf{MT2024107} & Shivam Padaliya \\
    \end{tabular}

    \vfill

    {\large \today\par}

\end{titlepage}

\thispagestyle{empty}
\newpage


% Abstract

\thispagestyle{empty}

\begin{abstractbox}
Modern web applications implement a variety of client-side validation rules using HTML5 attributes and JavaScript. These checks improve user experience but run entirely in the user's browser, which can be modified by an attacker. This project focuses exclusively on \textbf{client-side web application bypass testing} for a taxi booking system.

We implemented a simple but realistic taxi booking web application using Spring Boot and static HTML pages for registration, login, and booking. The frontend uses HTML5 validation features such as \texttt{type="email"}, \texttt{required}, \texttt{min}, \texttt{max}, and \texttt{readonly} to enforce rules like valid email format, passenger count limits, and future pickup time.

We then designed bypass test cases in two ways: (i) manual manipulation using browser DevTools, and (ii) automated manipulation using Selenium WebDriver in \texttt{BookingBypassUITest}. The automated tests disable form validation, remove or change HTML attributes, inject invalid values (past pickup times, negative or large passenger counts, and far-away coordinates), and submit the forms.

The results clearly show that once the client is considered untrusted, all client-side protections can be bypassed. This demonstrates that client-side validation cannot be treated as a security boundary and that all critical rules must be revalidated on the server. The project serves as a focused case study of the CSE731 topic ``Client-side web applications testing (bypass testing).''
\end{abstractbox}

\vspace{0.5cm}
\noindent\textbf{Keywords:} Client-side validation, Bypass testing, Web application security, Selenium, Spring Boot, HTML5

\newpage
\tableofcontents
\newpage


% 1. Introduction

\section{Introduction}

\subsection{Background and Motivation}

Most web applications validate user input at the browser level before sending it to the server. This client-side validation is fast and convenient, but it runs in an environment fully controlled by the user. An attacker can view and change any HTML, JavaScript, or HTTP request that leaves the browser.

If the backend assumes that the browser has already enforced all rules, then simple manipulations of the DOM or requests can break business constraints, corrupt data, or create unsafe states. Security guidelines such as OWASP therefore recommend that all validations be repeated and enforced on the server.

\subsection{Objective}

The objective of this project is:

\begin{quote}
To design and execute client-side bypass test cases for a taxi booking web application, using manual DevTools manipulation and Selenium automation, and to demonstrate that client-side validation alone is not sufficient for enforcing business rules.
\end{quote}

We focus on:

\begin{itemize}
    \item Identifying client-side validation rules in the taxi booking flow.
    \item Designing realistic bypass scenarios (invalid emails, past dates, extreme passenger counts).
    \item Implementing automated bypass tests in \texttt{BookingBypassUITest}.
    \item Relating the results to the CSE731 project category of client-side bypass testing.
\end{itemize}

\subsection{Scope}

The scope of this work is:

\begin{itemize}
    \item A Spring Boot web application with basic registration, login, and booking functionality.
    \item Static HTML pages with meaningful client-side validation rules.
    \item One Selenium-based automation class for bypass testing.
    \item Manual testing using browser DevTools.
\end{itemize}

We do not include a complete backend security audit, Postman/Newman collections, or a separate JUnit service-layer test suite in this version. The goal is to keep the system simple and focus on client-side bypass behavior.

\subsection{Course Alignment}

The project is based on the CSE731 project category:

\begin{quote}
Client-side web applications testing (bypass testing): Projects that involve testing of client side code of a web application by designing test cases that bypass client-side validation and sending changed/corrupt input to the server.
\end{quote}

Our work matches this description by:

\begin{itemize}
    \item Implementing a client-side validated UI for a realistic domain (taxi booking).
    \item Designing manual and automated tests that modify DOM state and validation attributes.
    \item Submitting changed or corrupt values to the backend from a compromised client.
\end{itemize}


% 2. Application Description

\section{Application Description}

\subsection{Taxi Booking Flow Overview}

The application, called \textbf{RideWithMe}, is a simplified taxi booking web application with three primary user-facing pages:

\begin{itemize}
    \item \textbf{Registration (\texttt{register.html})}: Create a new user account.
    \item \textbf{Login (\texttt{index.html})}: Authenticate and enter the system.
    \item \textbf{Dashboard (\texttt{pages/dashboard.html})}: Create a taxi booking by providing locations, pickup time, passenger count, and vehicle type.
\end{itemize}

A separate page \texttt{pages/bookings.html} displays the list of bookings for the logged-in user and is useful for verifying that a booking created through bypass actually reached the backend.

\subsection{Business Rules from the UI Perspective}

From a client-side point of view, the following business rules are expressed through HTML5 validation:

\begin{itemize}
    \item Email should be syntactically valid.
    \item Password should have a minimum length.
    \item Pickup time should be in the future.
    \item Passenger count should be within a limited range.
    \item Vehicle type should be selected from a predefined list.
\end{itemize}

These constraints are implemented using HTML attributes such as \texttt{type}, \texttt{required}, \texttt{min}, \texttt{max}, and \texttt{readonly}, and therefore can be targeted in bypass testing.


% 3. Client-Side Validation Rules

\section{Client-Side Validation Rules}

\subsection{Authentication Pages}

The registration form makes use of standard HTML5 validation features. A simplified excerpt is shown below:

\begin{lstlisting}[language=HTML, caption={Registration form with HTML5 validation}]
<form id="registerForm">
  <div class="form-group">
    <label for="email">Email address</label>
    <input type="email" id="email" name="email" required
           placeholder="you@email.com">
  </div>

  <div class="form-group">
    <label for="phoneNumber">Phone number</label>
    <input type="tel" id="phoneNumber" name="phoneNumber" required
           pattern="[0-9]{10}" placeholder="10-digit number">
  </div>

  <div class="form-group">
    <label for="password">Password</label>
    <input type="password" id="password" name="password" required
           minlength="6" placeholder="Minimum 6 characters">
  </div>

  <button type="submit" class="btn-primary">Create account</button>
</form>
\end{lstlisting}

The login form in \texttt{index.html} follows a similar pattern using \texttt{type="email"}, \texttt{required}, and \texttt{minlength} attributes.

\subsection{Booking Page}

On the booking page (\texttt{pages/dashboard.html}), the \texttt{bookingForm} includes fields for pickup and drop coordinates, pickup time, number of passengers, vehicle type, and payment mode. Typical client-side rules include:

\begin{itemize}
    \item Pickup time has a \texttt{min} attribute set to the current time.
    \item Passenger count has \texttt{min} and \texttt{max} attributes.
    \item The estimated fare field is marked as \texttt{readonly}.
    \item Vehicle type is marked as \texttt{required}.
\end{itemize}

All of these rules are enforced by the browser but can be changed at runtime.


% 4. System Architecture (Brief)

\section{System Architecture (Brief)}

\subsection{Technology Stack}

\begin{itemize}
    \item \textbf{Backend:} Spring Boot (Java)
    \item \textbf{Frontend:} Static HTML5, CSS3, vanilla JavaScript
    \item \textbf{Testing:} JUnit 5, Selenium WebDriver (Chrome, headless)
\end{itemize}

\subsection{Relevant File Structure}

\begin{verbatim}
RideWithMe/
├── src/
│   ├── main/
│   │   ├── java/com/taxiapp/...        (controllers, services, etc.)
│   │   └── resources/
│   │       └── static/
│   │           ├── index.html          (login)
│   │           ├── register.html       (registration)
│   │           └── pages/
│   │               ├── dashboard.html  (booking)
│   │               └── bookings.html   (view bookings)
│   └── test/
│       └── java/com/taxiapp/selenium/
│           └── BookingBypassUITest.java
\end{verbatim}

The Selenium test class uses \texttt{@SpringBootTest(webEnvironment = RANDOM\_PORT)} so that the browser interacts with a real HTTP server instance during testing.


% 5. Bypass Testing Methodology

\section{Bypass Testing Methodology}

\begin{keybox}[title=Testing Strategy]
\begin{enumerate}
    \item Identify client-side constraints in HTML and JavaScript.
    \item Design input values that violate those constraints.
    \item Apply these values using:
    \begin{itemize}
        \item Browser DevTools (manual).
        \item Selenium WebDriver with JavaScript execution (automated).
    \end{itemize}
    \item Observe browser behavior and confirm that constraints are bypassed.
\end{enumerate}
\end{keybox}

\subsection{Manual Bypass Using DevTools}

Manual testing uses the browser console to change DOM state. Examples include:

\begin{lstlisting}[language=JavaScript, caption={Passenger field manipulation}]
document.getElementById('passengers').value = 25;
\end{lstlisting}

\begin{lstlisting}[language=JavaScript, caption={Removing minimum time constraint}]
const t = document.getElementById('pickupTime');
t.removeAttribute('min');
t.value = '2020-01-01T10:00';
\end{lstlisting}

\begin{lstlisting}[language=JavaScript, caption={Editing readonly fare field}]
const f = document.getElementById('estimatedFare');
f.removeAttribute('readonly');
f.value = '10';
\end{lstlisting}

Form-level HTML5 validation can also be disabled:

\begin{lstlisting}[language=JavaScript, caption={Disabling HTML5 form validation}]
document.getElementById('registerForm')
        .setAttribute('novalidate', 'true');
\end{lstlisting}

\subsection{Automated Bypass Using Selenium}

Automated tests are implemented in \texttt{BookingBypassUITest}, which uses Selenium WebDriver together with \texttt{JavascriptExecutor}:

\begin{lstlisting}[language=Java, caption={Sample Selenium DOM manipulation}]
JavascriptExecutor js = (JavascriptExecutor) driver;

// Disable validation on booking form
js.executeScript(
  "document.getElementById('bookingForm')" +
  ".setAttribute('novalidate','true');"
);

// Remove min/max and set 25 passengers
js.executeScript(
  "var p = document.getElementById('passengers');" +
  "p.removeAttribute('min'); p.removeAttribute('max'); p.value = 25;"
);
\end{lstlisting}

After manipulating the DOM, the test submits the form:

\begin{lstlisting}[language=Java, caption={Submitting the manipulated form}]
WebElement form = driver.findElement(By.id("bookingForm"));
form.submit();
\end{lstlisting}


% 6. Selenium Test Suite

\section{Selenium Test Suite: \texttt{BookingBypassUITest}}

\subsection{Environment Setup}

\begin{itemize}
    \item \textbf{Spring Boot:} \texttt{@SpringBootTest(webEnvironment = RANDOM\_PORT)}
    \item \textbf{Browser:} Chrome in headless mode via \texttt{ChromeDriver}
    \item \textbf{Waits:} \texttt{WebDriverWait} with a 10-second timeout
    \item \textbf{Base URL:} Constructed from the injected random port
\end{itemize}

\subsection{Test Case Summary}

\begin{table}[h!]
\centering
\caption{Selenium bypass test cases in \texttt{BookingBypassUITest}}
\begin{tabular}{@{}p{0.27\textwidth}p{0.28\textwidth}p{0.33\textwidth}@{}}
\toprule
\textbf{Test Name} & \textbf{Scenario} & \textbf{Bypass Technique} \\ \midrule
\texttt{testRegisterAndLoginWithInvalidEmailBypassingValidation} &
Register and log in with invalid email format. &
Set \texttt{novalidate} on \texttt{registerForm} and \texttt{loginForm}, submit email ``seleniumtestcom''. \\[0.4em]
\texttt{testBookingWithExcessivePassengersBypassingValidation} &
Booking with \texttt{passengers = 25}. &
Remove \texttt{min}/\texttt{max} from \texttt{passengers}, set to 25, disable validation, submit. \\[0.4em]
\texttt{testBookingWithPastPickupTimeBypassingValidation} &
Booking with past pickup time. &
Set \texttt{pickupTime} to \texttt{2020-01-01T10:00} via JS, disable validation, submit. \\[0.4em]
\texttt{testBookingWithZeroPassengersBypassingValidation} &
Booking with \texttt{passengers = 0}. &
Remove \texttt{min} attribute, set value to 0, disable validation, submit. \\[0.4em]
\texttt{testBookingWithNegativePassengersBypassingValidation} &
Booking with \texttt{passengers = -5}. &
Remove \texttt{min}, set value to -5, disable validation, submit. \\[0.4em]
\texttt{testBookingWithOutsideServiceAreaBypassingValidation} &
Pickup in Mumbai, drop in Delhi. &
Set pickup and drop coordinates far apart, disable validation, submit. \\[0.4em]
\texttt{testEnableAndManipulateEstimatedFareField} &
Override readonly estimated fare. &
Remove \texttt{readonly} and set \texttt{estimatedFare = 10} via JS. \\[0.4em]
\texttt{testDisableClientSideValidationFlag} &
Disable validation on booking form. &
Set \texttt{novalidate} on \texttt{bookingForm} and verify attribute is present. \\ \bottomrule
\end{tabular}
\end{table}

\subsection{Example Test}

Listing~\ref{lst:excessive-passengers-test} shows an example bypass test for 25 passengers.

\begin{lstlisting}[language=Java, caption={Excessive passengers bypass test}, label={lst:excessive-passengers-test}]
@Test
@DisplayName("UI BYPASS: Submit booking with 25 passengers")
public void testBookingWithExcessivePassengersBypassingValidation() {
    driver.get(baseUrl + "/pages/dashboard.html");
    waitForElement("bookingForm");

    JavascriptExecutor js = (JavascriptExecutor) driver;

    js.executeScript(
        "document.getElementById('bookingForm')"
      + ".setAttribute('novalidate','true');"
    );

    js.executeScript(
        "var p = document.getElementById('passengers');"
      + "p.removeAttribute('min'); p.removeAttribute('max'); p.value = 25;"
    );

    js.executeScript("document.getElementById('pickupLat').value = 19.0760;");
    js.executeScript("document.getElementById('pickupLon').value = 72.8777;");
    js.executeScript("document.getElementById('dropLat').value = 19.1136;");
    js.executeScript("document.getElementById('dropLon').value = 72.9083;");
    js.executeScript("document.getElementById('pickupTime').value = '2030-01-01T10:00';");
    js.executeScript("document.getElementById('vehicleType').value = 'SEDAN';");
    js.executeScript("document.getElementById('paymentMode').value = 'CASH';");

    Assertions.assertEquals("25",
        driver.findElement(By.id("passengers")).getAttribute("value"));

    WebElement form = driver.findElement(By.id("bookingForm"));
    form.submit();

    wait.until(ExpectedConditions.or(
        ExpectedConditions.urlContains("dashboard"),
        ExpectedConditions.urlContains("bookings"),
        ExpectedConditions.presenceOfElementLocated(By.id("message"))
    ));
}
\end{lstlisting}


% 7. Execution Instructions

\section{Execution Instructions}

\subsection{Prerequisites}

\begin{itemize}
    \item Java 11 or higher
    \item Maven 3.6 or higher
    \item Google Chrome installed
    \item ChromeDriver compatible with installed Chrome
\end{itemize}

\subsection{Running the Application}

\begin{lstlisting}[language=bash]
cd RideWithMe
mvn spring-boot:run
\end{lstlisting}

The application is then accessible on the configured local port.

\subsection{Running Selenium Bypass Tests}

\begin{lstlisting}[language=bash]
mvn test -Dtest=BookingBypassUITest
\end{lstlisting}

This will start the Spring Boot application in a test context and execute all bypass scenarios defined in \texttt{BookingBypassUITest}.


% 8. Key Observations and Findings

\section{Key Observations and Findings}

\begin{keybox}[title=Client-Side Validation Weaknesses]
\begin{itemize}
    \item HTML5 attributes such as \texttt{min}, \texttt{max}, \texttt{required}, \texttt{readonly}, and \texttt{type} can be freely modified or removed.
    \item Form-level validation can be disabled using the \texttt{novalidate} attribute or by programmatic submission.
    \item Readonly fields such as estimated fare are not protected once the attacker can access the DOM.
    \item Once the client is considered untrusted, relying on browser checks for security is unsafe.
\end{itemize}
\end{keybox}

\subsection{Implications for Backend Design}

The main conclusion for backend design is:

\begin{quote}
Assuming that the browser has already validated input is unsafe. Every critical rule must be enforced again on the server, regardless of what the client claims to have done.
\end{quote}

If the server fails to validate data, the types of manipulations demonstrated here can result in bookings with invalid passenger counts, inconsistent times, and manipulated fare values.


% 9. Demonstration Plan

\section{Demonstration Plan}

\subsection{Normal User Flow}

\begin{enumerate}
    \item Register a new user on \texttt{register.html}.
    \item Log in via \texttt{index.html}.
    \item Create a booking on \texttt{pages/dashboard.html} using valid inputs.
    \item Show that the browser blocks obviously invalid inputs when no bypass is attempted.
\end{enumerate}

\subsection{Manual Bypass Demonstration}

\begin{enumerate}
    \item Open DevTools on the booking page.
    \item Modify \texttt{passengers}, \texttt{pickupTime}, and \texttt{estimatedFare} using JavaScript.
    \item Add \texttt{novalidate} to the booking form.
    \item Submit the form and show that the browser does not prevent submission.
\end{enumerate}

\subsection{Automated Bypass Demonstration}

\begin{enumerate}
    \item Run \texttt{mvn test -Dtest=BookingBypassUITest}.
    \item Present the test matrix for the eight Selenium cases.
    \item Explain how each test maps to a specific bypass scenario.
\end{enumerate}

% 10. Individual Contributions

\section{Individual Contributions}

\subsection*{MT2024106 Omkar Dhananjay Satav}

\begin{itemize}
    \item Design and implementation of the taxi booking backend.
    \item Creation of HTML flows for registration, login, and booking.
    \item Manual client-side bypass experiments using DevTools.
\end{itemize}

\subsection*{MT2024107 Shivam Padaliya}

\begin{itemize}
    \item Design of client-side validation rules in HTML.
    \item Implementation of Selenium-based bypass tests in \texttt{BookingBypassUITest}.
    \item Documentation and structuring of bypass scenarios for CSE731 requirements.
\end{itemize}

% 11. Conclusion

\section{Conclusion}

This project demonstrates, in a concrete and reproducible way, that:

\begin{enumerate}
    \item Client-side validation, even when carefully implemented, is not a security boundary because the browser is under attacker control.
    \item Simple DOM manipulations using DevTools or Selenium can bypass most visible protections, including required fields, numeric ranges, date constraints, and readonly flags.
    \item A dedicated client-side bypass testing effort is essential to understand attacker capabilities and to motivate robust server-side validation.
\end{enumerate}

By focusing strictly on client-side bypass testing of a taxi booking flow, the project satisfies the CSE731 requirement for ``Client-side web applications testing (bypass testing)'' and provides a compact example that can be extended with strong backend validation and additional security tests in future work.


\section*{References}

\begin{enumerate}
    \item CSE731 Software Testing course material, IIIT Bangalore.
    \item OWASP Web Security Testing Guide, Input Validation and Client-Side Security sections.
    \item Spring Boot Documentation: \url{https://spring.io/projects/spring-boot}
    \item Selenium WebDriver Documentation: \url{https://www.selenium.dev/documentation/}
    \item JUnit 5 User Guide: \url{https://junit.org/junit5/docs/current/user-guide/}
\end{enumerate}

\end{document}
